\newpage

\begin{center}
{\bf \large 关于本书}
\end{center}

为什么本书和其它的并行编程书籍不同呢?
主要原因在于我们集中在实现层面,包括:

\begin{itemize}

\item 这里几乎没有理论内容,诸如 O() 分析、最大理论加速、PRAM、有向无环图(DAG)等等;

\item Real code is featured throughout.

\item We use the main parallel platforms---OpenMP, CUDA and MPI---rather
than languages that at this stage are largely experimental or arcane.

\item The running {\it performance} themes---communications latency,
memory/network contention, load balancing and so on---are interleaved
throughout the book, discussed in the context of specific platforms or
applications.

\item Considerable attention is paid to techniques for debugging.

\end{itemize}

The main programming language used is C/C++, but some of the code is in
R, which has become the pre-eminent language for data analysis.  As a
scripting language, R can be used for rapid prototyping.  In our case
here, it enables me to write examples which are much less less cluttered
than they would be in C/C++, thus easier for students to discern the
fundamental parallelixation principles involved.  For the same reason,
it makes it easier for students to write their own parallel code,
focusing on those principles.  And R has a rich set of parallel
libraries.

It is assumed that the student is reasonably adept in programming, and
has math background through linear algebra.  An appendix reviews the
parts of the latter needed for this book.  Another appendix presents an
overview of various systems issues that arise, such as process
scheduling and virtual memory.

It should be note that most of the code examples in the book are NOT
optimized.  The primary emphasis is on simplicity and clarity of the
techniques and languages used.  However, there is plenty of discussion
on factors that affect speed, such cache coherency issues, network
delays, GPU memory structures and so on.

Here's how to get the code files you'll see in this book: The book is
set in LaTeX, and the raw {\bf .tex} files are available in
\url{http://heather.cs.ucdavis.edu/~matloff/158/PLN}.  Simply download
the relevant file (the file names should be clear), then use a text
editor to trim to the program code of interest.

In order to illustrate for students the fact that research and teaching
(should) enhance each other, I occasionally will make brief references
here to some of my research work.

Like all my open source textbooks, this one is {\bf constantly
evolving}.  I continue to add new topics, new examples and so on, and of
course fix bugs and improve the exposition.  For that reason, it is
better to link to the latest version, which will always be at
\url{http://heather.cs.ucdavis.edu/~matloff/158/PLN/ParProcBook.pdf},
rather than to copy it.

For that reason, feedback is highly appreciated.  I wish to thank Stuart
Ambler, Matt Butner, Stuart Hansen, Bill Hsu, Sameer Khan, Mikel
McDaniel, Richard Minner, Lars Seeman, Marc Sosnick,
and Johan Wikstr{\"o}m for their
comments.  I'm also very grateful to Professor Hsu for his making
available to me advanced GPU-equipped machines.

You may also be interested in my open source textbook on probability and
statistics, at \url{http://heather.cs.ucdavis.edu/probstatbook}.

This work is licensed under a Creative Commons Attribution-No Derivative
Works 3.0 United States License. Copyright is retained by N. Matloff in
all non-U.S. jurisdictions, but permission to use these materials in
teaching is still granted, provided the authorship and licensing
information here is displayed in each unit. I would appreciate being
notified if you use this book for teaching, just so that I know the
materials are being put to use, but this is not required.


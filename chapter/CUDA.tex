\chapter{GPU 计算简介——}
\label{chap:cuda} 

即使你不玩视频游戏,你可能也要感谢这些游戏玩家,
因为正是人数众多的他们让一种高效的并行运算设备——{\bf 图形处理单元}(GPUs),成为可能。
是的,你的程序可以直接在你电脑中的显卡上运行,
即使你的程序不涉及图形处理或游戏。

\section{概述}

The video game market is so lucrative that the industry has developed
ever-faster GPUs, in order to handle ever-faster and ever-more visually
detailed video games.  These actually are parallel processing hardware
devices, so around 2003 some people began to wonder if one might use
them for parallel processing of nongraphics applications.

Originally this was cumbersome.  One needed to figure out clever ways of
mapping one's application to some kind of graphics problem, i.e. ways
to disguising one's problem so that it appeared to be doing graphics
computations.  Though some high-level interfaces were developed to
automate this transformation, effective coding required some
understanding of graphics principles.

But current-generation GPUs separate out the graphics operations, and
now consist of multiprocessor elements that run under the familiar
shared-memory threads model.  Thus they are easily programmable.
Granted, effective coding still requires an intimate knowledge of the
hardwre, but at least it's (more or less) familiar hardware, not
requiring knowledge of graphics.

Moreover, unlike a multicore machine, with the ability to run just a few
threads at one time, e.g. four threads on a quad core machine, GPUs can
run {\it  hundreds or thousands} of threads at once.  There are various
restrictions that come with this, but you can see that there is
fantastic potential for speed here.

NVIDIA has developed the CUDA language as a vehicle for programming on
their GPUs.  It's basically just a slight extension of C, and has become
very popular.  More recently, the OpenCL language has been developed by
Apple, AMD and others (including NVIDIA).  It too is a slight extension
of C, and it aims to provide a uniform interface that works with
multicore machines in addition to GPUs.  OpenCL is not yet in as broad
use as CUDA, so our discussion here focuses on CUDA and NVIDIA GPUs.

Also, the discussion will focus on NVIDIA's Tesla line.  This then led
to the second generation, Fermi, and then Kepler.  Unless otherwise
stated, all statements here refer to Tesla.

Some terminology: 

\begin{itemize}

\item A CUDA program consists of code to be run on the {\bf host}, i.e.
the CPU, and code to run on the {\bf device}, i.e. the GPU.

\item A function that is called by the host to execute on the device is
called a {\bf kernel}.

\item Threads in an application are grouped into {\bf blocks}.  The
entirety of blocks is called the {\bf grid} of that application.  

\end{itemize}

\section{Example:  Calculate Row Sums}

Here's a sample program.  And I've kept the sample simple:  It just
finds the sums of all the rows of a matrix.  

\begin{lstlisting}[numbers=left]
#include <stdio.h>
#include <stdlib.h>
#include <cuda.h>

// CUDA example:  finds row sums of an integer matrix m

// find1elt() finds the rowsum of one row of the nxn matrix m, storing the
// result in the corresponding position in the rowsum array rs; matrix
// stored as 1-dimensional, row-major order

__global__ void find1elt(int *m, int *rs, int n)
{
   int rownum = blockIdx.x;  // this thread will handle row # rownum
   int sum = 0;
   for (int k = 0; k < n; k++)
      sum += m[rownum*n+k];
   rs[rownum] = sum;
}

int main(int argc, char **argv)
{
    int n = atoi(argv[1]);  // number of matrix rows/cols
    int *hm, // host matrix
        *dm, // device matrix
        *hrs, // host rowsums
        *drs; // device rowsums
    int msize = n * n * sizeof(int);  // size of matrix in bytes
    // allocate space for host matrix
    hm = (int *) malloc(msize);  
    // as a test, fill matrix with consecutive integers
    int t = 0,i,j;
    for (i = 0; i < n; i++) {
       for (j = 0; j < n; j++) {
          hm[i*n+j] = t++;
       }
    }
    // allocate space for device matrix 
    cudaMalloc((void **)&dm,msize);
    // copy host matrix to device matrix
    cudaMemcpy(dm,hm,msize,cudaMemcpyHostToDevice);
    // allocate host, device rowsum arrays
    int rssize = n * sizeof(int);
    hrs = (int *) malloc(rssize);  
    cudaMalloc((void **)&drs,rssize);
    // set up parameters for threads structure
    dim3 dimGrid(n,1);  // n blocks 
    dim3 dimBlock(1,1,1);  // 1 thread per block
    // invoke the kernel
    find1elt<<<dimGrid,dimBlock>>>(dm,drs,n);
    // wait for kernel to finish
    cudaThreadSynchronize();
    // copy row vector from device to host
    cudaMemcpy(hrs,drs,rssize,cudaMemcpyDeviceToHost);
    // check results
    if (n < 10) for(int i=0; i<n; i++) printf("%d\n",hrs[i]);
    // clean up
    free(hm);
    cudaFree(dm);
    free(hrs);
    cudaFree(drs);
}
\end{lstlisting}

This is mostly C, with a bit of CUDA added here and there.  Here's
how the program works:

\begin{itemize}

\item Our {\bf main()} runs on the host.  

\item Kernel functions are identified by {\bf \_\_global\_\_ void}.
They are called by the host and run on the device, thus serving as
entries to the device.  

We have only one kernel invocation here, but could have many, say with
the output of one serving as input to the next.

\item Other functions that will run on the device, called by functions
running on the device, must be identified by {\bf \_\_device\_\_}, e.g.

\begin{Verbatim}[fontsize=\relsize{-2}]
__device__ int sumvector(float *x, int n)
\end{Verbatim}

Note that unlike kernel functions, device functions can have return
values, e.g. {\bf int} above.

\item When a kernel is called, each thread runs it.  Each thread
receives the same arguments. 

% \item The kernel may call other functions to run on the device, 
% using data stored in the device memory.  Such functions may in turn make
% such calls.

\item Each block and thread has an ID, stored in programmer-accessible 
structs {\bf blockIdx} and {\bf threadIdx}.  We'll discuss the 
details later, but for now, we'll just note that here the statement 

\begin{Verbatim}[fontsize=\relsize{-2}]
int rownum = blockIdx.x; 
\end{Verbatim}

picks up the block number, which our code in this example uses to
determine which row to sum.

\item One calls {\bf cudaMalloc()} on the host to dynamically allocate
space on the device's memory.\footnote{This function 
cannot be called from the device itself.   However, {\bf malloc()} is
available from the device, and device memory allocated by it can be
copied to the host.  See the NVIDIA programming guide for details.}
Execution of the statement

\begin{Verbatim}[fontsize=\relsize{-2}]
cudaMalloc((void **)&drs,rssize);
\end{Verbatim}

allocates space {\it on the device}, pointed to by {\bf drs}, a variable
in the {\it host's} address space.

The space allocated by a {\bf cudaMalloc()} call on the device is global
to all kernels, and resides in the global memory of the device (details
on memory types later).

One can also allocate device memory statically.  For example, the
statement

\begin{Verbatim}[fontsize=\relsize{-2}]
__device int z[100];
\end{Verbatim}

appearing outside any function definition would allocate space on device
global memory, with scope global to all kernels.  However, it is not
accessible to the host.

\item Data is transferred to and from the host and device memories via
{\bf cudaMemcpy()}.  The fourth argument specifies the direction, e.g.
cudaMemcpyHostToDevice, cudaMemcpyDeviceToHost or
cudaMemcpyDeviceToDevice.

\item Kernels return {\bf void} values, so values are returned
via a kernel's arguments.  

\item Device functions (which we don't have here) can return values.
They are called only by kernel functions or other device functions.

\item Note carefully that a call to the kernel doesn't block; it returns
immediately.  For that reason, the code above has a host barrier call,
to avoid copying the results back to the host from the device before
they're ready:

\begin{Verbatim}[fontsize=\relsize{-2}]
cudaThreadSynchronize();
\end{Verbatim}

On the other hand, if our code were to have another kernel call, say on
the next line after 

\begin{Verbatim}[fontsize=\relsize{-2}]
find1elt<<<dimGrid,dimBlock>>>(dm,drs,n);
\end{Verbatim}

and if some of the second call's input arguments were the outputs of the
first call, there would be an implied barrier betwwen the two calls; the
second would not start execution before the first finished.

Calls like {\bf cudaMemcpy()} do block until the operation completes.

There is also a thread barrier available for the threads themselves, at
the block level.  The call is

\begin{Verbatim}[fontsize=\relsize{-2}]
__syncthreads();
\end{Verbatim}

This can only be invoked by threads within a block, not across blocks.
In other words, this is barrier synchronization within blocks.

\item I've written the program so that each thread will handle one row
of the matrix.  I've chosen to store the matrix in one-dimensional form
in row-major order, and the matrix is of size n x n, so the loop

\begin{Verbatim}[fontsize=\relsize{-2}]
for (int k = 0; k < n; k++)
   sum += m[rownum*n+k];
\end{Verbatim}

will indeed traverse the n elements of row number {\bf rownum}, and
compute their sum.  That sum is then placed in the proper element of the
output array:

\begin{Verbatim}[fontsize=\relsize{-2}]
rs[rownum] = sum;
\end{Verbatim}

\item After the kernel returns, the host must copy the result back from the
device memory to the host memory, in order to access the results of the
call.

\end{itemize}

\section{Understanding the Hardware Structure}

{\it Scorecards, get your scorecards here!  You can't tell the players
without a scorecard}---classic cry of vendors at baseball games

{\it Know thy enemy}---Sun Tzu, {\it The Art of War}

The enormous computational potential of GPUs cannot be unlocked without
an intimate understanding of the hardware.  This of course is a 
fundamental truism in the parallel processing world, but it is acutely
important for GPU programming.  This section presents an overview of the
hardware. 

\subsection{Processing Units}

A GPU consists of a large set of {\bf streaming multiprocessors} (SMs).
Since each SM is essentially a multicore machine in its own right, you
might say the GPU is a multi-multiprocessor machine.  

Each SM consists of a number of {\bf streaming processors} (SPs),
individual cores.  The cores run threads, as with ordinary cores, but
threads in an SM run in lockstep, to be explained below.

It is important to understand the motivation for this SM/SP hierarchy:
Two threads located in different SMs cannot synchronize with each other
in the barrier sense.  Though this sounds like a negative at first, it
is actually a great advantage, as the independence of threads in
separate SMs means that the hardware can run faster.  So, if the CUDA
application programmer can write his/her algorithm so as to have certain
independent chunks, and those chunks can be assigned to different SMs
(we'll see how, shortly), then that's a ``win.''

Note that at present, word size is 32 bits.  Thus for instance
floating-point operations in hardware were originally in single precision
only, though newer devices are capable of double precision.

\subsection{Thread Operation}

GPU operation is highly threaded, and again, understanding of the
details of thread operation is key to good performance.

\subsubsection{SIMT Architecture}

When you write a CUDA application program, you partition the threads
into groups called {\bf blocks}.  The hardware will assign an entire
block to a single SM, though several blocks can run in the same SM.  The
hardware will then divide a block into {\bf warps}, 32 threads to a
warp.  Knowing that the hardware works this way, the programmer controls
the block size and the number of blocks, and in general writes the code
to take advantage of how the hardware works.  

The central point is that {\it all the threads in a warp run the code in
lockstep}.  During the machine instruction fetch cycle, the same
instruction will be fetched for all of the threads in the warp.  Then in
the execution cycle, each thread will either execute that particular
instruction or execute nothing.   The execute-nothing case occurs in the
case of branches; see below.  This is the classical {\bf single
instruction, multiple data} (SIMD) pattern used in some early
special-purpose computers such as the ILLIAC; here it is called {\bf
single instruction, multiple thread} (SIMT).

The syntactic details of grid and block configuration will be presented
in Section \ref{threadshierarchy}.

\subsubsection{The Problem of Thread Divergence}

{\it The SIMT nature of thread execution has major implications for
performance.}  Consider what happens with if/then/else code.  If some
threads in a warp take the ``then'' branch and others go in the ``else''
direction, they cannot operate in lockstep.  That means that some
threads must wait while others execute.  This renders the code at that
point serial rather than parallel, a situation called {\bf thread
divergence}.  As one CUDA Web tutorial points out, this can be a
``performance killer.''  (On the other hand, threads in the same block
but in different warps can diverge with no problem.)

\subsubsection{``OS in Hardware''}

Each SM runs the threads on a timesharing basis, just like an operating
system (OS).  This timesharing is implemented in the hardware, though,
not in software as in the OS case.  

The ``hardware OS'' runs largely in analogy with an ordinary OS:

\begin{itemize}

\item A process in an ordinary OS is given a fixed-length timeslice, so
that processes take turns running. In a GPU's hardware OS, warps take
turns running, with fixed-length timeslices.

\item With an ordinary OS, if a process reaches an input/output
operation, the OS suspends the process while I/O is pending, even if its
turn is not up.  The OS then runs some other process instead, so as to
avoid wasting CPU cycles during the long period of time needed for the
I/O.  

With an SM, though, the analogous situation occurs when there is a long
memory operation, to global memory; if a a warp of threads needs to
access global memory (including local memory; see below), the SM will
schedule some other warp while the memory access is pending.

\end{itemize}

The hardware support for threads is extremely good; a context switch
takes very little time, quite a contrast to the OS case.  Moreover, as
noted above, the long latency of global memory may be solvable by having
a lot of threads that the hardware can timeshare to hide that latency;
while one warp is fetching data from memory, another warp can be
executing, thus not losing time due to the long fetch delay.  For these
reasons, CUDA programmers typically employ a large number of threads,
each of which does only a small amount of work---again, quite a contrast
to something like OpenMP.

\subsection{Memory Structure}

The GPU memory hierarchy plays a key role in performance.  Let's discuss
the most important two types of memory first---shared and global.  

\subsubsection{Shared and Global Memory}

Here is a summary:

\begin{tabular}{|r|r|r|}
\hline
type & shared & global \\ \hline 
\hline
scope & glbl. to block  & glbl. to app. \\ \hline 
size & small & large \\ \hline 
location & on-chip & off-chip \\ \hline 
speed & blinding & molasses \\ \hline
lifetime & kernel & application \\ \hline 
host access? & no & yes \\ \hline 
cached? & no & no \\ \hline 
\end{tabular}

In prose form:

\begin{itemize}

\item Shared memory:  All the threads in an SM share this memory, and
use it to communicate among themselves, just as is the case with threads
in CPUs.  Access is very fast, as this memory is on-chip.  It is
declared inside the kernel, or in the kernel call (details below).

On the other hand, shared memory is small, currently 16K bytes per SM,
and the data stored in it are valid only for the life of the
currently-executing kernel.  Also, shared memory cannot be accessed
by the host.

\item Global memory:  This is shared by all the threads in an entire
application, and is persistent across kernel calls, throughout the
life of the application, i.e. until the program running on the host
exits.  It is usually much larger than shared memory.  It is accessible
from the host.  Pointers to global memory can (but do not have to) be
declared outside the kernel. 

On the other hand, global memory is off-chip and very slow, taking
hundreds of clock cycles per access instead of just a few.  As noted
earlier, this can be ameliorated by exploiting latency hiding; we will
elaborate on this in Section \ref{coal}.

\end{itemize}

The reader should pause here and reread the above comparison between
shared and global memories.  {\it The key implication is that shared
memory is used essentially as a programmer-managed cache.} Data will
start out in global memory, but if a variable is to be accessed multiple
times by the GPU code, it's probably better for the programmer to
write code that copies it to shared memory, and then access the copy
instead of the original.  If the variable is changed and is to be
eventually transmitted back to the host, the programmer must include
code to copy it back to global memory.

Accesses to global and shared memory are done via half-warps, i.e. an
attempt is made to do all memory accesses in a half-warp simultaneously.
In that sense, only threads in a half-warp run simultaneously, but the
full warp is {\it scheduled} to run contemporaneously by the hardware
OS, first one half-warp and then the other.

The host can access global memory via {\bf cudaMemcpy()}, as seen
earlier.  It cannot access shared memory.  Here is a typical pattern:

\begin{Verbatim}[fontsize=\relsize{-2}]
__global__ void abckernel(int *abcglobalmem)
{
   __shared__ int abcsharedmem[100];
   // ... code to copy some of abcglobalmem to some of abcsharedmem
   // ... code for computation
   // ... code to copy some of abcsharedmem to some of abcglobalmem
}
\end{Verbatim}

Typically you would write the code so that each thread deals with its
own portion of the shared data, e.g. its own portion of {\bf
abcsharedmem} and {\bf abcglobalmem} above.  However, all the threads in
that block can read/write any element in {\bf abcsharedmem}. 

Shared memory consistency (recall Section \ref{consistency}) is
sequential within a thread, but {\bf relaxed} among threads in a block:
A write by one thread is not guaranteed to be visible to the others in a
block until {\bf \_\_syncthreads()} is called.  On the other hand,
writes by a thread {\it will} be visible to that same thread in
subsequent reads without calling {\bf \_\_syncthreads()}.  Among the
implications of this is that if each thread writes only to portions of
shared memory that are not read by other threads in the block, then {\bf
\_\_syncthreads()} need not be called.

In the code fragment above, we allocated the shared memory through a
C-style declaration:

\begin{Verbatim}[fontsize=\relsize{-2}]
__shared__ int abcsharedmem[100];
\end{Verbatim}

It is also possible to allocate shared memory in the kernel call, along
with the block and thread configuration.  Here is an example:

\begin{lstlisting}[numbers=left]
#include <stdio.h>
#include <stdlib.h>
#include <cuda.h>

// CUDA example:  illustrates kernel-allocated shared memory; does
// nothing useful, just copying an array from host to device global,
// then to device shared, doubling it there, then copying back to device
// global then host

__global__ void doubleit(int *dv, int n)
{  extern __shared__ int sv[];
   int me = threadIdx.x;  
   // threads share in copying dv to sv, with each thread copying one
   // element
   sv[me] = 2 * dv[me];
   dv[me] = sv[me];
}

int main(int argc, char **argv)
{
    int n = atoi(argv[1]);  // number of matrix rows/cols
    int *hv, // host array
        *dv; // device array
    int vsize = n * sizeof(int);  // size of array in bytes
    // allocate space for host array
    hv = (int *) malloc(vsize);  
    // fill test array with consecutive integers
    int t = 0,i;
    for (i = 0; i < n; i++) 
       hv[i] = t++;
    // allocate space for device array 
    cudaMalloc((void **)&dv,vsize);
    // copy host array to device array
    cudaMemcpy(dv,hv,vsize,cudaMemcpyHostToDevice);
    // set up parameters for threads structure
    dim3 dimGrid(1,1);
    dim3 dimBlock(n,1,1);  // all n threads in the same block
    // invoke the kernel; third argument is amount of shared memory
    doubleit<<<dimGrid,dimBlock,vsize>>>(dv,n);
    // wait for kernel to finish
    cudaThreadSynchronize();
    // copy row array from device to host
    cudaMemcpy(hv,dv,vsize,cudaMemcpyDeviceToHost);
    // check results
    if (n < 10) for(int i=0; i<n; i++) printf("%d\n",hv[i]);
    // clean up
    free(hv);
    cudaFree(dv);
}
\end{lstlisting}

Here the variable {\bf sv} is kernel allocated.  It's declared in
the statement

\begin{Verbatim}[fontsize=\relsize{-2}]
extern __shared__ int sv[];
\end{Verbatim}

but actually allocated during the kernel invocation

\begin{Verbatim}[fontsize=\relsize{-2}]
doubleit<<<dimGrid,dimBlock,vsize>>>(dv,n);
\end{Verbatim}

in that third argument within the chevrons, {\bf vsize}.

Note that one can only directly declare one region of space in this
manner.  This has two implications:

\begin{itemize}

\item Suppose we have two {\bf \_\_device\_\_} functions, each declared
an {\bf extern \_\_shared\_\_} array like this.  Those two arrays will
occupy the same place in memory!

\item Suppose within one {\bf \_\_device\_\_} function, we wish to have
two {\bf extern \_\_shared\_\_} arrays.  We cannot do that literally,
but we can share the space via subarrays, e.g.:

\begin{Verbatim}[fontsize=\relsize{-2}]
int *x = &sv[120];
\end{Verbatim}

would set up {\bf x} as a subarray of {\bf sv} above, starting at
element 120.

\end{itemize}

One can also set up shared arrays of fixed length in the same code.
Declare them before the variable-length one.  

In our example above, the array {\bf sv} is syntactically local to the
function {\bf doubleit()}, but is shared by all invocations of that
function in the block, thus acting ``global'' to them in a sense.  But
the point is that it is not accessible from within {\it other} functions
running in that block.  In order to achieve the latter situation, a
shared array can be declared outside any function.  

\subsubsection{Global-Memory Performance Issues}
\label{coal}

As noted, the latency (Section \ref{latencybandwidth}) for global memory
is quite high, on the order of hundreds of clock cycles.  However, the
hardware attempts to ameliorate this problem in a couple of ways.

First, as mentioned earlier, if a warp has requested a global memory
access that will take a long time, the hardware will schedule another
warp to run while the first is waiting for the memory access to
complete.  This is an example of a common parallel processing technique
called {\bf latency hiding}.

Second, the bandwidth (Section \ref{latencybandwidth}) to global memory
can be high, due to hardware actions called {\bf coalescing}.  This
simply means that if the hardware sees that the threads in this
half-warp (or at least the ones currently accessing global memory) are
accessing consecutive words, the hardware can execute the memory
requests in groups of up to 32 words at a time.  This works because the
memory is low-order interleaved (Section \ref{interleaving}), and is
true for both reads and writes.

The newer GPUs go even further, coalescing much more general access
patterns, not just to consecutive words.

The programmer may be able to take advantage of coalescing, by a
judicious choice of algorithms and/or by inserting padding into arrays
(Section \ref{bankclash}).

% Global memory is organized into {\bf partitions} of 256 bytes each.
% There will be six or eight partitions, depending on the GPU model.

\subsubsection{Shared-Memory Performance Issues}

Shared memory is divided into banks, in a low-order interleaved
manner (recall Section \ref{banks}):  Words with consecutive addresses
are stored in consecutive banks, mod the number of banks, i.e.  wrapping
back to 0 when hitting the last bank.  If for instance there are 8
banks, addresses 0, 8, 16,... will be in bank 0, addresses 1, 9, 17,...
will be in bank 1 and so on.  (Actually, older devices have 16 banks,
while newer ones have 32.) The fact that all memory accesses in a
half-warp are attempted simultaneously implies that the best access to
shared memory arises when the accesses are to different banks, just as
for the case of global memory.  

An exception occurs in {\bf broadcast}.  If all threads in the block
wish to read from the same word in the same bank, the word will be sent
to all the requestors simultaneously without conflict.  However, if only
some theads try to read the same word, there may or may not be a
conflict, as the hardware chooses a bank for broadcast in some
unspecified way.

As in the discussion of global memory above, we should write our code to
take advantage of these structures.

The biggest performance issue with shared memory is its size, as little
as 16K per SM in many GPU cards.  And remember, this is divvied up among
the blocks on a given SM.  If we have 4 blocks running on an SM, each
one can only use 16K/4 = 4K bytes of shared memory.

\subsubsection{Host/Device Memory Transfer Performance Issues}

Copying data between host and device can be a major bottleneck.  One way
to ameliorate this is to use {\bf cudaMallocHost()} instead of {\bf
malloc()} when allocating memory on the host.  This sets up page-locked
memory, meaning that it cannot be swapped out by the OS' virtual memory
system.  This allows the use of DMA hardware to do the memory copy, 
said to make {\bf cudaMemcpy()} twice as fast.

\subsubsection{Other Types of Memory}
\label{othermems}

There are also other types of memory.  Again, let's start with a
summary:

\begin{tabular}{|r|r|r|r|r|}
\hline
type & registers & local & constant & texture \\ \hline 
\hline
scope & single thread & single thread & glbl. to app. & glbl. to app.\\ \hline 
location & device & device & host+device cache & host+device cache\\ \hline 
speed & fast & molasses & fast if cache hit & fast if cache hit \\ \hline
lifetime & kernel & kernel & application & application \\ \hline 
host access? & no & no & yes & yes \\ \hline 
device access? & read/write & read/write & read & read \\ \hline 
\end{tabular}

\begin{itemize}

\item {\bf Registers:}  

Each SM has a set of registers, much more numerous than in a CPU.
Access to them is very fast, said to be slightly faster than to shared
memory.

The compiler normally stores the local variables for a device function
in registers, but there are exceptions.  An array won't be placed in
registers if the array is too large, or if the array has variable index
values, such as

\begin{Verbatim}[fontsize=\relsize{-2}]
int z[20],i;
...
y = z[i];
\end{Verbatim}

Since registers are not indexable by the hardware, the compiler cannot
allocate {\bf z} to registers in this case.  If on the other hand, the
only code accessing {\bf z} has constant indices, e.g. {\bf z[8]}, the
compiler may put {\bf z} in registers.

\item {\bf Local memory:}  

This is physically part of global memory, but is an area within that
memory that is allocated by the compiler for a given thread.  As such,
it is slow, and accessible only by that thread.  The compiler allocates
this memory for local variables in a device function if the compiler
cannot store them in registers.  This is called {\bf register spill}.

\item {\bf Constant memory:}  

As the name implies, it's read-only from the device (read/write by the
host), for storing values that will not be changed by device code.  It
is off-chip, thus potentially slow, but has a cache on the chip.  At
present, the size is 64K.

One designates this memory with {\bf \_\_constant\_\_}, as a global
variable in the source file.  One sets its contents from the host via
{\bf cudaMemcpyToSymbol()}, whose (simple form for the) call is

\begin{Verbatim}[fontsize=\relsize{-2}]
cudaMemcpyToSymbol(var_name,pointer_to_source,number_bytes_copy,cudaMemcpyHostToDevice)
\end{Verbatim}

For example:

\begin{Verbatim}[fontsize=\relsize{-2}]
__constant__ int x;  // not contained in any function

// host code
int y = 3;
cudaMemcpyToSymbol("x",&y,sizeof(int));
...

// device code
int z;
z = x;
\end{Verbatim}

Note again that the name Constant refers to the fact that device code
cannot change it.  But host code certainly can change it between kernel
calls.  This might be useful in iterative algorithms like this:

\begin{Verbatim}[fontsize=\relsize{-2}]
/ host code
for 1 to number of iterations
   set Constant array x
   call kernel (do scatter op)
   cudaThreadSynchronize()
   do gather op, using kernel results to form new x

// device code
use x together with thread-specific data
return results to host
\end{Verbatim}

\item {\bf Texture:}

This is similar to constant memory, in the sense that it is read-only
and cached.  The difference is that the caching is two-dimensional.  The
elements {\bf a[i][j]} and {\bf a[i+1][j]} are far from each other in
the global memory, but since they are ``close'' in a two-dimensional
sense, they may reside in the same cache line.

\end{itemize}

\subsection{Threads Hierarchy}
\label{threadshierarchy}

Following the hardware, threads in CUDA software follow a hierarchy:

\begin{itemize}

\item The entirety of threads for an application is called a {\bf grid}.  

\item A grid consists of one or more {\bf blocks} of threads.

\item Each block has its own ID within the grid, consisting of an ``x
coordinate'' and a ``y coordinate.''  

\item Likewise each thread has x, y and z coordinates within whichever
block it belongs to.  

\item Just as an ordinary CPU thread needs to be able to sense its ID, e.g.
by calling {\bf omp\_get\_thread\_num()} in OpenMP, CUDA threads need
to do the same.  A CUDA thread can access its block ID via the built-in
variables {\bf blockIdx.x} and {\bf blockIdx.y}, and can access its
thread ID within its block via {\bf threadIdx.x}, {\bf threadIdx.y} and
{\bf threadIdx.z}.

\item The programmer specifies the grid size (the numbers of rows and
columns of blocks within a grid) and the block size (numbers of rows,
columns and layers of threads within a block).  In the first example
above, this was done by the code

\begin{Verbatim}[fontsize=\relsize{-2}]
dim3 dimGrid(n,1);
dim3 dimBlock(1,1,1);
find1elt<<<dimGrid,dimBlock>>>(dm,drs,n);
\end{Verbatim}

Here the grid is specified to consist of n ($n \times 1$) blocks,
and each block consists of just one ($1 \times 1 \times 1)$ thread.

That last line is of course the call to the kernel.  As you can see,
CUDA extends C syntax to allow specifying the grid and block sizes.
CUDA will store this information in structs of type {\bf dim3}, in this
case our variables {\bf gridDim} and {\bf blockDim}, accessible to the
programmer, again with member variables for the various dimensions, e.g.
{\bf blockDim.x} for the size of the X dimension for the number of
threads per block.

\item All threads in a block run in the same SM, though more than
one block might be on the same SM.

\item The ``coordinates'' of a block within the grid, and of a thread
within a block, are merely abstractions.  If for instance one is
programming computation of heat flow across a two-dimensional slab, the
programmer may find it clearer to use two-dimensional IDs for the
threads.  {\it But this does not correspond to any physical arrangement
in the hardware.}

\end{itemize}

As noted, the motivation for the two-dimensional block arrangment is to
make coding conceptually simpler for the programmer if he/she is working
an application that is two-dimensional in nature.  

For example, in a matrix application one's parallel algorithm might be
based on partitioning the matrix into rectangular submatrices ({\bf
tiles}), as we'll do in Section \ref{partitioned}.  In a small example
there, the matrix

\begin{equation}
A = 
\left (
\begin{array}{ccc}
1 & 5 & 12 \\
0 & 3 & 6 \\
4 & 8 & 2
\end{array}
\right )
\end{equation}

is partitioned as 

\begin{equation}
A =
\left (
\begin{array}{cc}
A_{00} & A_{01} \\
A_{10} & A_{11}
\end{array}
\right ) ,
\end{equation}

where

\begin{equation}
A_{00} =
\left (
\begin{array}{cc}
1 & 5 \\
0 & 3
\end{array}
\right )  ,
\end{equation}

\begin{equation}
A_{01} =
\left (
\begin{array}{cc}
12 \\
6 
\end{array}
\right ),
\end{equation}

\begin{equation}
A_{10} =
\left (
\begin{array}{cc}
4 & 8 
\end{array}
\right )  
\end{equation}

and

\begin{equation}
A_{11} =
\left (
\begin{array}{c}
2
\end{array}
\right ).
\end{equation}

We might then have one block of threads handle $A_{00}$, another block
handle $A_{01}$ and so on.  CUDA's two-dimensional ID system for blocks
makes life easier for programmers in such situations.  

\subsection{What's NOT There}

{\it We're not in Kansas anymore, Toto}---character Dorothy Gale in {\it
The Wizard of Oz}

It looks like C, it feels like C, and for the most part, it {\it is} C.
But in many ways, it's quite different from what you're used to:

\begin{itemize}

\item You don't have access to the C library, e.g. {\bf printf()} (the
library consists of host machine language, after all).  There are
special versions of math functions, however, e.g. {\bf \_\_sin()}.

\item No recursion.

\item No stack.  Functions are essentially inlined, rather than their 
calls being handled by pushes onto a stack.

\item No pointers to functions.

\end{itemize}

\section{Synchronization, Within and Between Blocks}

As mentioned earlier, a barrier for the threads in the same block is
available by calling {\bf \_\_syncthreads()}.  Note carefully that if
one thread writes a variable to shared memory and another then reads
that variable, one must call this function (from both threads) in order
to get the latest value.  Keep in mind that within a block, different
warps will run at different times, making synchronization vital.

Remember too that  threads across blocks cannot sync with each other in
this manner.  There are, though, several {\bf atomic}
operations---read/modify/write actions that a thread can execute without
{\bf pre-emption}, i.e. without interruption---available on both global
and shared memory.  For example, {\bf atomicAdd()} performs a
fetch-and-add operation, as described in Section \ref{fanda} of this
book.  The call is

\begin{Verbatim}[fontsize=\relsize{-2}]
atomicAdd(address of integer variable,inc);
\end{Verbatim}

where {\bf address of integer variable} is the address of the (device)
variable to add to, and {\bf inc} is the amount to be added.  The return
value of the function is the value originally at that address before the
operation.

There are also {\bf atomicExch()} (exchange the two operands), {\bf
atomicCAS()} (if the first operand equals the second, replace the first
by the third), {\bf atomicMin()}, {\bf atomicMax()}, {\bf atomicAnd()},
{\bf atomicOr()}, and so on.

Use {\bf -arch=sm\_11} when compiling, e.g.

\begin{Verbatim}[fontsize=\relsize{-2}]
nvcc -g -G yoursrc.cu -arch=sm_11
\end{Verbatim}

Though a barrier could in principle be constructed from the atomic
operations, its overhead would be quite high.  In earlier models that
was near a microsecond, and though that problem has been ameliorated in
more recent models, implementing a barrier in this manner .  would not
be not much faster than attaining interblock synchronization by
returning to the host and calling {\bf cudaThreadSynchronize()} there.
Recall that the latter {\it is} a possible way to implement a barrier,
since global memory stays intact in between kernel calls, but again, it
would be slow.

So, what if synchronization is really needed?  This is the case, for
instance, for iterative algorithms, where all threads must wait at the
end of each iteration.

If you have a small problem, maybe you can get satisfactory performance
by using just one block.  You'll have to use a larger granularity, i.e.
more work assigned to each thread.  But using just one block means
you're using only one SM, thus only a fraction of the potential power of
the machine.  

If you use multiple blocks, though, your only feasible option for
synchronization is to rely on returns to the host, where synchronization
occurs via {\bf cudaThreadSynchronize()}.  You would then have the
situation outlined in the discussion of Constant memory in Section
\ref{othermems}.

\section{More on the Blocks/Threads Tradeoff}

Resource size considerations must be kept in mind when you design your
code and your grid configuration.  In particular, note the following:

   \begin{itemize}

   \item Each block in your code is assigned to some SM.  It will be
   tied to that SM during the entire execution of your kernel, though of
   course it will not constantly be running during that time. 

   \item If there are more blocks than can be accommodated by all the
   SMs, then some blocks will need to wait for assignment; when a block
   finishes, that block's resources, e.g. shared memory, can now be
   assigned to a waiting block.

   \item The programmer has no control over which block is assigned to
   which SM.

   \item Within a block, threads execute by the warp, 32 threads.  At
   any give time, the SM is running one warp, chosen by the GPU OS.
   
   \item The GPU has a limit on the number of threads that can run on a
   single block, typically 512, and on the total number of threads running  
   on an SM, 786.

   \item If a block contains fewer than 32 threads, only part of the
   processing power of the SM it's running on will be used.  So block
   size should normally be at least 32.  Moreover, for the same reason,
   block size should ideally be a multiple of 32. 

   \item If your code makes used of shared memory, or does within-block
   synchronization, the larger the block size, the better.

   \item We want to use the full power of the GPU, with its many SMs,
   thus implying a need to use at least as many blocks as there are SMs
   (which may require smaller blocks).

   \item Moreover, due to the need for latency hiding in memory access,
   we want to have lots of warps, so that some will run while others
   are doing memory access.

   \item Two threads doing unrelated work, or the same work but with
   many if/elses, would cause a lot of thread divergence if they were in
   the same block.

   \item A commonly-cited rule of thumb is to have between 128 and 256
   threads per block.

   \end{itemize}

Though there is a limit on the number of blocks, this limit will be much
larger than the number of SMs.  So, you may have multiple blocks running
on the same SM.  Since execution is scheduled by the warp anyway, there
appears to be no particular drawback to having more than one block on
the same SM.

\section{Hardware Requirements, Installation, Compilation, Debugging}

You do need a suitable NVIDIA video card.  There is a list at
\url{http://www.nvidia.com/object/cuda_gpus.html}; see also the
Wikipedia entry, \url{http://en.wikipedia.org/wiki/CUDA#Supported_GPUs}.
If you have a Linux system, run {\bf lspci} to determine what kind you
have.\footnote{It might be in {\bf /sbin}.}

Download the CUDA toolkit from NVIDIA.  Just plug ``CUDA download'' into
a Web search engine to find the site.  Install as directed.

You'll need to set your search and library paths to include the CUDA {\bf
bin} and {\bf lib} directories.

To compile {\bf x.cu} (and yes, use the {\bf .cu} suffix), type

\begin{Verbatim}[fontsize=\relsize{-2}]
$ nvcc -g -G x.cu 
\end{Verbatim}

The {\bf -g -G} options are for setting up debugging, the first for host
code, the second for device code.  You may also need to specify

\begin{Verbatim}[fontsize=\relsize{-2}]
-I/your_CUDA_include_path
\end{Verbatim}

to pick up the file {\bf cuda.h}.  Run the code as you normally would.

You may need to take special action to set your library path properly.
For example, on Linux machines, set the environment variable {\bf
LD\_LIBRARY\_PATH} to include the CUDA library.

To determine the limits, e.g. maximum number of threads, for your
device, use code like this:

\begin{Verbatim}[fontsize=\relsize{-2}]
cudaDeviceProp Props;
cudaGetDeviceProperties(&Props,0);
\end{Verbatim}

The 0 is for device 0, assuming you only have one device.  The return
value of {\bf cudaGetDeviceProperties()} is a complex C struct
whose components are listed at
\url{http://developer.download.nvidia.com/compute/cuda/2_3/toolkit/docs/online/group__CUDART__DEVICE_g5aa4f47938af8276f08074d09b7d520c.html}.

Here's a simple program to check some of the properties of device 0:

\begin{lstlisting}[numbers=left]
#include<cuda.h>
#include <stdio.h>

int main()
{
   cudaDeviceProp Props;
   cudaGetDeviceProperties( &Props,0);

   printf("shared mem: %d)\n", Props.sharedMemPerBlock);
   printf("max threads/block: %d\n",Props.maxThreadsPerBlock);
   printf("max blocks: %d\n",Props.maxGridSize[0]);
   printf("total Const mem: %d\n",Props.totalConstMem);
}
\end{lstlisting}

Under older versions of CUDA, such as 2.3, one can debug using GDB as
usual.  You must compile your program in emulation mode, using the {\bf
-deviceemu} command-line option.  This is no longer available as of
version 3.2.  CUDA also includes a special version of GDB, CUDA-GDB
(invoked as {\bf cuda-gdb}) for real-time debugging.  However, on
Unix-family platforms it runs only if X11 is not running.  Short of
dedicating a machine for debugging, you may find it useful to install a
version 2.3 in addition to the most recent one to use for debugging.

\section{Example:  Improving the Row Sums Program}

The issues involving coalescing in Section \ref{coal} would suggest that
our rowsum code might run faster with column sums, to take advantage of
the memory banking.  (So the user would either need to take the
transpose first, or have his code set up so that the matrix is in
transpose form to begin with.) As two threads in the same half-warp
march down adjoining columns in lockstep, they will always be accessing
adjoining words in memory.  

So, I modified the program accordingly (not shown), and compiled the two
versions, as {\bf rs} and {\bf cs}, the row- and column-sum versions of
the code, respectively.

This did produce a small improvement (confirmed in subsequent runs,
needed in any timing experiment):

\begin{Verbatim}[fontsize=\relsize{-2}]
pc5:~/CUDA% time rs 20000
2.585u 1.753s 0:04.54 95.3%     0+0k 7104+0io 54pf+0w
pc5:~/CUDA% time cs 20000
2.518u 1.814s 0:04.40 98.1%     0+0k 536+0io 5pf+0w
\end{Verbatim}

But let's compare it to a version running only on the CPU,

\begin{lstlisting}[numbers=left]
#include <stdio.h>
#include <stdlib.h>

// non-CUDA example:  finds col sums of an integer matrix m

// find1elt() finds the colsum of one col of the nxn matrix m, storing the
// result in the corresponding position in the colsum array cs; matrix
// stored as 1-dimensional, row-major order

void find1elt(int *m, int *cs, int n)
{
   int sum=0;
   int topofcol;
   int col,k;
   for (col = 0; col < n; col++) {
      topofcol = col;
      sum = 0;
      for (k = 0; k < n; k++)
         sum += m[topofcol+k*n];
      cs[col] = sum;
   }
}

int main(int argc, char **argv)
{
    int n = atoi(argv[1]);  // number of matrix cols/cols
    int *hm, // host matrix
        *hcs; // host colsums
    int msize = n * n * sizeof(int);  // size of matrix in bytes
    // allocate space for host matrix
    hm = (int *) malloc(msize);  
    // as a test, fill matrix with consecutive integers
    int t = 0,i,j;
    for (i = 0; i < n; i++) {
       for (j = 0; j < n; j++) {
          hm[i*n+j] = t++;
       }
    }
    int cssize = n * sizeof(int);
    hcs = (int *) malloc(cssize);  
    find1elt(hm,hcs,n);
    if (n < 10) for(i=0; i<n; i++) printf("%d\n",hcs[i]);
    // clean up
    free(hm);
    free(hcs);
}
\end{lstlisting}

How fast does this non-CUDA version run?

\begin{Verbatim}[fontsize=\relsize{-2}]
pc5:~/CUDA% time csc 20000
61.110u 1.719s 1:02.86 99.9%    0+0k 0+0io 0pf+0w
\end{Verbatim}

Very impressive!  No wonder people talk of CUDA in terms like ``a
supercomputer on our desktop.''  And remember, this includes the time to
copy the matrix from the host to the device (and to copy the output
array back).  And we didn't even try to optimize thread configuration,
memory coalescing and bank usage, making good use of memory hierarchy,
etc.\footnote{Neither has the CPU-only version of the program been
optimized.  As pointed out by Bill Hsu, the row-major version of that
program should run faster than the column-major one, due to cache
consideration.}

On the other hand, remember that this is an ``embarrassingly parallel''
application, and in many applications we may have to settle for a much
more modest increase, and work harder to get it.

\section{Example:  Finding the Mean Number of Mutual Outlinks}

As in Sections \ref{mutlinks} and \ref{ompmutlinks}, consider a network
graph of some kind, such as Web links.  For any two vertices, say any
two Web sites, we might be interested in mutual outlinks, i.e. outbound
links that are common to two Web sites.  The CUDA code below finds the
mean number of mutual outlinks, among all pairs of sites in a set of Web
sites.  

\begin{lstlisting}[numbers=left]
#include <cuda.h>
#include <stdio.h>

// CUDA example:  finds mean number of mutual outlinks, among all pairs
// of Web sites in our set; in checking all (i,j) pairs, thread k will
// handle all i such that i mod totth = k, where totth is the number of
// threads

// procpairs() processes all pairs for a given thread
__global__ void procpairs(int *m, int *tot, int n)
{  int totth = gridDim.x * blockDim.x,  // total number of threads
       me = blockIdx.x * blockDim.x + threadIdx.x;  // my thread number
   int i,j,k,sum = 0; 
   for (i = me; i < n; i += totth) {  // do various rows i
      for (j = i+1; j < n; j++) {  // do all rows j > i
         for (k = 0; k < n; k++)
            sum += m[n*i+k] * m[n*j+k];
      }
   }
   atomicAdd(tot,sum);
}

int main(int argc, char **argv)
{  int n = atoi(argv[1]),  // number of vertices
       nblk = atoi(argv[2]);  // number of blocks
    int *hm, // host matrix
        *dm, // device matrix
        htot, // host grand total
        *dtot; // device grand total
    int msize = n * n * sizeof(int);  // size of matrix in bytes
    // allocate space for host matrix
    hm = (int *) malloc(msize);  
    // as a test, fill matrix with random 1s and 0s
    int i,j;
    for (i = 0; i < n; i++) {
       hm[n*i+i] = 0;
       for (j = 0; j < n; j++) {
          if (j != i) hm[i*n+j] = rand() % 2;
       }
    }
    // allocate space for device matrix 
    cudaMalloc((void **)&dm,msize);
    // copy host matrix to device matrix
    cudaMemcpy(dm,hm,msize,cudaMemcpyHostToDevice);
    htot = 0;
    // set up device total and initialize it
    cudaMalloc((void **)&dtot,sizeof(int));
    cudaMemcpy(dtot,&htot,sizeof(int),cudaMemcpyHostToDevice);
    // set up parameters for threads structure
    dim3 dimGrid(nblk,1);
    dim3 dimBlock(192,1,1);
    // invoke the kernel
    procpairs<<<dimGrid,dimBlock>>>(dm,dtot,n);
    // wait for kernel to finish
    cudaThreadSynchronize();
    // copy total from device to host
    cudaMemcpy(&htot,dtot,sizeof(int),cudaMemcpyDeviceToHost);
    // check results
    if (n <= 15) {
       for (i = 0; i < n; i++) {
          for (j = 0; j < n; j++) 
             printf("%d ",hm[n*i+j]);
          printf("\n");
       }
    }
    printf("mean = %f\n",htot/float((n*(n-1))/2));
    // clean up
    free(hm);
    cudaFree(dm);
    cudaFree(dtot);
}
\end{lstlisting}

Again we've used the method in Section \ref{mutlinks} to partition the
various pairs (i,j) to the different threads.  Note the use of {\bf
atomicAdd()}.

The above code is hardly optimal.  The reader is encouraged to find
improvements.

\section{Example:  Finding Prime Numbers}
\label{primefinder}

The code below finds all the prime numbers from 2 to {\bf n}.

\begin{lstlisting}[numbers=left]
#include <stdio.h>
#include <stdlib.h>
#include <cuda.h>

// CUDA example:  illustration of shared memory allocation at run time;
// finds primes using classical Sieve of Erathosthenes:  make list of
// numbers 2 to n, then cross out all multiples of 2 (but not 2 itself),
// then all multiples of 3, etc.; whatever is left over is prime; in our
// array, 1 will mean "not crossed out" and 0 will mean "crossed out"

// IMPORTANT NOTE: uses shared memory, in a single block, without
// rotating parts of array in and out of shared memory; thus limited to
// n <= 4000 if have 16K shared memory

// initialize sprimes, 1s for the odds, 0s for the evens; see sieve()
// for the nature of the arguments
__device__ void initsp(int *sprimes, int n, int nth, int me) 
{
   int chunk,startsetsp,endsetsp,val,i;
   sprimes[2] = 1;
   // determine sprimes chunk for this thread to init
   chunk = (n-1) / nth;
   startsetsp = 2 + me*chunk;
   if (me < nth-1) endsetsp = startsetsp + chunk - 1;
   else endsetsp = n;
   // now do the init
   val = startsetsp % 2; 
   for (i = startsetsp; i <= endsetsp; i++) {
      sprimes[i] = val;
      val = 1 - val;
   }
   // make sure sprimes up to date for all
   __syncthreads();
}

// copy sprimes back to device global memory; see sieve() for the nature
// of the arguments
__device__ void cpytoglb(int *dprimes, int *sprimes, int n, int nth, int me) 
{
   int startcpy,endcpy,chunk,i;
   chunk = (n-1) / nth;
   startcpy = 2 + me*chunk;
   if (me < nth-1) endcpy = startcpy + chunk - 1;
   else endcpy = n;
   for (i = startcpy; i <= endcpy; i++) dprimes[i] = sprimes[i];
   __syncthreads();
}

// finds primes from 2 to n, storing the information in dprimes, with
// dprimes[i] being 1 if i is prime, 0 if composite; nth is the number
// of threads (threadDim somehow not recognized)
__global__ void sieve(int *dprimes, int n, int nth)
{
   extern __shared__ int sprimes[];
   int me = threadIdx.x;  
   int nth1 = nth - 1;
   // initialize sprimes array, 1s for odds, 0 for evens
   initsp(sprimes,n,nth,me);
   // "cross out" multiples of various numbers m, with each thread doing
   // a chunk of m's; always check first to determine whether m has
   // already been found to be composite; finish when m*m > n
   int maxmult,m,startmult,endmult,chunk,i;
   for (m = 3; m*m <= n; m++) {
      if (sprimes[m] != 0) {
         // find largest multiple of m that is <= n
         maxmult = n / m;
         // now partition 2,3,...,maxmult among the threads
         chunk = (maxmult - 1) / nth;
         startmult = 2 + me*chunk;
         if (me < nth1) endmult = startmult + chunk - 1;
         else endmult = maxmult;
      }
      // OK, cross out my chunk
      for (i = startmult; i <= endmult; i++) sprimes[i*m] = 0;
   }
   __syncthreads();
   // copy back to device global memory for return to host
   cpytoglb(dprimes,sprimes,n,nth,me);
}

int main(int argc, char **argv)
{
    int n = atoi(argv[1]),  // will find primes among 1,...,n
        nth = atoi(argv[2]);  // number of threads
    int *hprimes,  // host primes list
        *dprimes;  // device primes list
    int psize = (n+1) * sizeof(int);  // size of primes lists in bytes
    // allocate space for host list
    hprimes = (int *) malloc(psize);  
    // allocate space for device list 
    cudaMalloc((void **)&dprimes,psize);
    dim3 dimGrid(1,1);
    dim3 dimBlock(nth,1,1);
    // invoke the kernel, including a request to allocate shared memory
    sieve<<<dimGrid,dimBlock,psize>>>(dprimes,n,nth);
    // check whether we asked for too much shared memory
    cudaError_t err = cudaGetLastError();
    if(err != cudaSuccess) printf("%s\n",cudaGetErrorString(err));
    // wait for kernel to finish
    cudaThreadSynchronize();
    // copy list from device to host
    cudaMemcpy(hprimes,dprimes,psize,cudaMemcpyDeviceToHost);
    // check results
    if (n <= 1000) for(int i=2; i<=n; i++) 
       if (hprimes[i] == 1) printf("%d\n",i);
    // clean up
    free(hprimes);
    cudaFree(dprimes);
}
\end{lstlisting}

This code has been designed with some thought as to memory speed and
thread divergence.  Ideally, we would like to use device shared memory
if possible, and to exploit the lockstep, SIMD nature of the hardware.

The code uses the classical Sieve of Erathosthenes, ``crossing out''
multiples of 2, 3, 5, 7 and so on to get rid of all the composite
numbers.  However, the code here differs from that in Section
\ref{sharedview}, even though both programs use the Sieve of
Erathosthenes.  

Say we have just two threads, A and B.  In the earlier version, thread A
might cross out all multiples of 19 while B handles multiples of 23.  In
this new version, thread A deals with only some multiples of 19 and B
handles the others for 19.  Then they both handle their own portions of
multiples of 23, and so on.  The thinking here is that the second
version will be more amenable to lockstep execution, thus causing less
thread divergence.

Thus in this new version, each thread handles a chunk of multiples of
the given prime.  Note the contrast of this with many CUDA examples, in
which each thread does only a small amount of work, such as computing a
single element in the product of two matrices.

In order to enhance memory performance, this code uses device shared
memory.  All the ``crossing out'' is done in the shared memory array
{\bf sprimes}, and then when we are all done, that is copied to the
device global memory array {\bf dprimes}, which is in turn copies to
host memory.  By the way, note that the amount of shared memory here is
determined dynamically.

However, device shared memory consists only of 16K bytes, which would
limit us here to values of {\bf n} up to about 4000.  Moreover, by using
just one block, we are only using a small part of the CPU.  Extending
the program to work for larger values of {\bf n} would require some
careful planning if we still wish to use shared memory.  

\section{Example:  Finding Cumulative Sums}
\label{cumulsums}

Here we wish to compute cumulative sums.  For instance, if the original
array is (3,1,2,0,3,0,1,2), then it is changed to (3,4,6,6,9,9,10,12).

(Note:  This is a special case of the {\it prefix scan} problem, covered
in Chapter \ref{chapter:prefix}.)

The general plan is for each thread to operate on one
chunk of the array.  A thread will find cumulative sums for its chunk,
and then adjust them based on the high values of the chunks that precede
it.  In the above example, for instance, say we have 4 threads.  The
threads will first produce (3,4), (2,2), (3,3) and (1,3).  Since thread
0 found a cumulative sum of 4 in the end, we must add 4 to each element
of (2,2), yielding (6,6).  Thread 1 had found a cumulative sum of 2 in
the end, which together with the 4 found by thread 0 makes 6.  Thus
thread 2 must add 6 to each of its elements, i.e. add 6 to (3,3),
yielding (9,9).  The case of thread 3 is similar.

Below is code for the special case of a single block:

\begin{lstlisting}[numbers=left]
// for this simple illustration, it is assumed that the code runs in
// just one block, and that the number of threads evenly divides n

// improvements that could be made:
//    1.  change to multiple blocks, to try to use all SMs
//    2.  possibly use shared memory
//    3.  have each thread work on staggered elements of dx, rather than
//        on contiguous ones, to get more efficient bank access

#include <cuda.h>
#include <stdio.h>

__global__ void cumulker(int *dx, int n)
{  
   int me = threadIdx.x;
   int csize  = n / blockDim.x;
   int start = me * csize;
   int i,j,base;
   for (i = 1; i < csize; i++) {
      j = start + i;
      dx[j] = dx[j-1] + dx[j];
   }
   __syncthreads();
   if (me > 0) {
      base = 0;
      for (j = 0; j < me; j++)
         base += dx[(j+1)*csize-1];
   }
   __syncthreads();
   if (me > 0) {
      for (i = start; i < start + csize; i++)
         dx[i] += base;
   }
}
\end{lstlisting}

\section{When Is it Advantageous to Use Shared Memory}

Shared memory only helps if we are doing multiple accesses to the data.
If for instance our code does a single read and a single write to an
element of an array, then transferring it back and forth between global
and shared memory isn't worthwhile.

Would the cumulative-sums program in Section \ref{cumulsums} benefit
from the use of shared memory?  (Put aside the fact that the code runs
in just one block, making use of just a sliver of the machine.)  The
answer appears to be that a modest imrprovement might be obtained.  Each
thread (except the first) reads many elements of {\bf dx} twice, some of
them three times.  There are also writes.

The case of the prime-finder program in Section \ref{primefinder} is
less clear, and probably quite dependent on whether we are using the
more advanced GPUs, which feature at least some L1 cache space.

\section{Example:  Transforming an Adjacency Matrix}
\label{transgraph1}

Here is a CUDA version of the code in Section \ref{transgraph}.

\begin{lstlisting}[numbers=left]
// CUDA example

// takes a graph adjacency matrix for a directed graph, and converts it
// to a 2-column matrix of pairs (i,j), meaning an edge from vertex i to
// vertex j; the output matrix must be in lexicographical order

// not claimed efficient, either in speed or in memory usage

#include <cuda.h>
#include <stdio.h>

// needs -lrt link flag for C++
#include <time.h>
float timediff(struct timespec t1, struct timespec t2)
{  if (t1.tv_nsec > t2.tv_nsec) {
         t2.tv_sec -= 1;
         t2.tv_nsec += 1000000000;
   }
   return t2.tv_sec-t1.tv_sec + 0.000000001 * (t2.tv_nsec-t1.tv_nsec);
}


// kernel transgraph() does this work
// arguments:
//    adjm:  the adjacency matrix (NOT assumed symmetric), 1 for edge, 0
//           otherwise; note: matrix is overwritten by the function
//    n:  number of rows and columns of adjm
//    adjmout:  output matrix
//    nout:  number of rows in adjmout

__global__ void tgkernel1(int *dadjm, int n, int *dcounts)
{  int tot1s,j;
   int me = blockDim.x * blockIdx.x + threadIdx.x;
   tot1s = 0;
   for (j = 0; j < n; j++) {
      if (dadjm[n*me+j] == 1) {
         dadjm[n*me+tot1s++] = j;
      }
   dcounts[me] = tot1s;
   }
}

__global__ void tgkernel2(int *dadjm, int n, 
   int *dcounts, int *dstarts, int *doutm)
{  int outrow,num1si,j;
   // int me = threadIdx.x;
   int me = blockDim.x * blockIdx.x + threadIdx.x;
   // fill in this thread's portion of doutm
   outrow = dstarts[me];
   num1si = dcounts[me];
   if (num1si > 0) {
      for (j = 0; j < num1si; j++) {
         doutm[2*outrow+2*j] = me;
         doutm[2*outrow+2*j+1] = dadjm[n*me+j];
     }
   }
}

// replaces counts by cumulative counts
void cumulcounts(int *c, int *s, int n)
{  int i;
   s[0] = 0;
   for (i = 1; i < n; i++) {
      s[i] = s[i-1] + c[i-1];
   }
}

int *transgraph(int *hadjm, int n, int *nout, int gsize, int bsize)
{  int *dadjm;  // device adjacency matrix
   int *houtm;  // host output matrix
   int *doutm;  // device output matrix
   int *hcounts;  // host counts vector
   int *dcounts;  // device counts vector
   int *hstarts;  // host starts vector
   int *dstarts;  // device starts vector
   hcounts = (int *) malloc(n*sizeof(int));
   hstarts = (int *) malloc(n*sizeof(int));
   cudaMalloc((void **)&dadjm,n*n*sizeof(int));
   cudaMalloc((void **)&dcounts,n*sizeof(int));
   cudaMalloc((void **)&dstarts,n*sizeof(int));
   houtm = (int *) malloc(n*n*sizeof(int));
   cudaMalloc((void **)&doutm,n*n*sizeof(int));
   cudaMemcpy(dadjm,hadjm,n*n*sizeof(int),cudaMemcpyHostToDevice);
   dim3 dimGrid(gsize,1);
   dim3 dimBlock(bsize,1,1);
   // calculate counts and starts first
   tgkernel1<<<dimGrid,dimBlock>>>(dadjm,n,dcounts);
   // cudaMemcpy(hadjm,dadjm,n*n*sizeof(int),cudaMemcpyDeviceToHost);
   cudaMemcpy(hcounts,dcounts,n*sizeof(int),cudaMemcpyDeviceToHost);
   cumulcounts(hcounts,hstarts,n);
   *nout = hstarts[n-1] + hcounts[n-1];
   cudaMemcpy(dstarts,hstarts,n*sizeof(int),cudaMemcpyHostToDevice);
   tgkernel2<<<dimGrid,dimBlock>>>(dadjm,n,dcounts,dstarts,doutm);
   cudaMemcpy(houtm,doutm,2*(*nout)*sizeof(int),cudaMemcpyDeviceToHost);
   free(hcounts);
   free(hstarts);
   cudaFree(dadjm);
   cudaFree(dcounts);
   cudaFree(dstarts);
   return houtm;
}

int main(int argc, char **argv)
{  int i,j;
   int *adjm;  // host adjacency matrix 
   int *outm;  // host output matrix
   int n = atoi(argv[1]);
   int gsize = atoi(argv[2]);
   int bsize = atoi(argv[3]);
   int nout;
   adjm = (int *) malloc(n*n*sizeof(int));
   for (i = 0; i < n; i++)
      for (j = 0; j < n; j++)
         if (i == j) adjm[n*i+j] = 0;
         else adjm[n*i+j] = rand() % 2;
   if (n < 10) {
      printf("adjacency matrix: \n");
      for (i = 0; i < n; i++) {
         for (j = 0; j < n; j++) printf("%d ",adjm[n*i+j]);
         printf("\n");
      }
   }
   
   struct timespec bgn,nd;
   clock_gettime(CLOCK_REALTIME, &bgn);

   outm = transgraph(adjm,n,&nout,gsize,bsize);
   printf("num rows in out matrix = %d\n",nout);
   if (nout < 50) {
      printf("out matrix: \n");
      for (i = 0; i < nout; i++) 
         printf("%d %d\n",outm[2*i],outm[2*i+1]);
   }

   clock_gettime(CLOCK_REALTIME, &nd);
   printf("%f\n",timediff(bgn,nd));
}

\end{lstlisting}

\section{Error Checking}

Every CUDA call (except for kernel invocations) returns an error code of
type {\bf cudaError\_t}.  One can view the nature of the error by
calling {\bf cudaGetErrorString()} and printing its output. 

For kernel invocations, one can call {\bf cudaGetLastError()}, which
does what its name implies.  A call would typically have the form

\begin{Verbatim}[fontsize=\relsize{-2}]
cudaError_t err = cudaGetLastError();
if(err != cudaSuccess) printf("%s\n",cudaGetErrorString(err));
\end{Verbatim}

You may also wish to {\bf cutilSafeCall()}, which is used by wrapping your 
regular CUDA call.  It automatically prints out error messages as above.

Each CUBLAS call returns a potential error code, of type {\bf
cublasStatus}, not checked here.

\section{Loop Unrolling}

{\bf Loop unrolling} is an old technique used on uniprocessor machines
to achieve speedup due to branch elimination and the like.  Branches
make it difficult to do instruction or data prefetching, so eliminating
them may speed things up.

The CUDA compiler provides the programmer with the {\bf unroll} pragma
to request loop unrolling.  Here an n-iteration {\bf for} loop is
changed to k copies of the body of the loop, each working on about n/k
iterations.  If n and k are known constant, GPU registers can be used to
implement the unrolled loop.

For example, the loop

\begin{Verbatim}[fontsize=\relsize{-2}]
for (i = 0; i < 2; i++ {
   sum += x[i];
   sum2 += x[i]*x[i];
}
\end{Verbatim}

could be unrolled to

\begin{Verbatim}[fontsize=\relsize{-2}]
sum += x[1];
sum2 += x[1]*x[1];
sum += x[2];
sum2 += x[2]*x[2];
\end{Verbatim}

Here n = k = 2.  If {\bf x} is local to this function, then unrolling
will allow the compiler to store it in a register, which could be a
great performance enhancer.

The compiler will try to do loop unrolling even if the programmer
doesn't request it, but the programmer can try to control things by
using the pragma:

\begin{Verbatim}[fontsize=\relsize{-2}]
#pragma unroll k
\end{Verbatim}

suggest to the compiler a k-fold unrolling.  Setting k = 1 will instruct
the compiler {\it not} to unroll.

\section{Short Vectors}

In CUDA, there are types such as {\bf int4}, {\bf char2} and so on, up
to four elements each.  So, an {\bf uint4} type is a set of four unsigned
{\bf int}s.  These are called {\bf short vectors}.

The key point is that a short vector can be treated as a single word in
terms of memory access and GPU instructions.  It may be possible to
reduce time by a factor of 4 by dividing arrays into chunks of four
contiguous words and making short vectors from them.

\section{The New Generation}

The latest GPU architecture from NVIDIA is called Kepler.  Many of the
advances are of the ``bigger and faster than before'' type.  These are
important, but be sure to note the significant architectural changes,
including:

\begin{itemize}

\item Host memory, device global memory and device shared memory share a
unifed address space.

\item On-chip memory can be apportioned to both shared memory and cache
memory.  Since shared memory is in essence a programmer-managed cache,
this gives the programmer access to a real cache, a great convenience to
the programmer though with a possible sacrifice in speed.  Note by the
way that this cache is aimed at spatial locality, not temporal
locality.

\end{itemize}

\section{CUDA from a Higher Level}

CUDA programming can involve a lot of work, and one is never sure that
one's code is fully efficient.  Fortunately, a number of libraries of
tight code have been developed for operations that arise often in
parallel programming.  

You are of course using CUDA code at the bottom, but without explicit
kernel calls.  And again, remember, the contents of device global memory
are persistent across kernel calls in the same application.  Therefore
you can mix explicit CUDA code and calls to these libraries.  Your
program might have multiple kernel invocations, some CUDA and others to
the libraries, with each using data in device global memory that was
written by earlier kernels.  In some cases, you may need to do a
conversion to get the proper type.

These packages can be deceptively simple.  {\bf Remember, each call to a
function in these packages involves a CUDA kernel call---with the
associated overhead.}

Programming in these libraries is typically much more convenient than in
direct CUDA.  Note, though, that even though these libraries have been
highly optimized for what they are intended to do, they will not
generally give you the fastest possible code for any given CUDA
application.  

We'll discuss a few such libraries in this section.

\subsection{CUBLAS}
\label{cublas}

CUDA includes some parallel linear algebra routines callable from
straight C code.  In other words, you can get the benefit of GPU in
linear algebra contexts without directly programming in CUDA.

\subsubsection{Example:  Row Sums Once Again}

Below is an example {\bf RowSumsCB.c}, the matrix row sums example
again, this time using CUBLAS.  We can find the vector of row sums of
the matrix A by post-multiplying A by a column vector of all 1s.

I compiled the code by typing

\begin{Verbatim}[fontsize=\relsize{-2}]
gcc -g -I/usr/local/cuda/include -L/usr/local/cuda/lib RowSumsCB.c -lcublas -lcudart
\end{Verbatim}

You should modify for your own CUDA locations accordingly.  Users who
merely wish to use CUBLAS will find the above more convenient, but if
you are mixing CUDA and CUBLAS, you would use {\bf nvcc}:

\begin{Verbatim}[fontsize=\relsize{-2}]
nvcc -g -G RowSumsCB.c -lcublas 
\end{Verbatim}

Here is the code:

\begin{lstlisting}[numbers=left]
#include <stdio.h>
#include <cublas.h>  // required include

int main(int argc, char **argv)
{
   int n = atoi(argv[1]);  // number of matrix rows/cols
   float *hm, // host matrix
         *hrs, // host rowsums vector
         *ones,  // 1s vector for multiply
         *dm, // device matrix
         *drs; // device rowsums vector
   // allocate space on host 
   hm = (float *) malloc(n*n*sizeof(float));
   hrs = (float *) malloc(n*sizeof(float));
   ones = (float *) malloc(n*sizeof(float));
   // as a test, fill hm with consecutive integers, but in column-major
   // order for CUBLAS; also put 1s in ones
   int i,j;
   float t = 0.0;
   for (i = 0; i < n; i++) {
      ones[i] = 1.0;
      for (j = 0; j < n; j++) 
         hm[j*n+i] = t++;
   }
   cublasInit();  // required init
   // set up space on the device
   cublasAlloc(n*n,sizeof(float),(void**)&dm);
   cublasAlloc(n,sizeof(float),(void**)&drs);
   // copy data from host to device
   cublasSetMatrix(n,n,sizeof(float),hm,n,dm,n);
   cublasSetVector(n,sizeof(float),ones,1,drs,1);
   // matrix times vector ("mv")
   cublasSgemv('n',n,n,1.0,dm,n,drs,1,0.0,drs,1);
   // copy result back to host
   cublasGetVector(n,sizeof(float),drs,1,hrs,1);
   // check results
   if (n < 20) for (i = 0; i < n; i++) printf("%f\n",hrs[i]);
   // clean up on device (should call free() on host too)
   cublasFree(dm);
   cublasFree(drs);
   cublasShutdown();
}
\end{lstlisting}

As noted in the comments, CUBLAS assumes FORTRAN-style, i.e.
column-major order, for matrices.

Now that you know the basic format of CUDA calls, the CUBLAS versions
will look similar.  In the call

\begin{Verbatim}[fontsize=\relsize{-2}]
cublasAlloc(n*n,sizeof(float),(void**)&dm);
\end{Verbatim}

for instance, we are allocating space on the device for an n x n
matrix of {\bf float}s.

The call 

\begin{Verbatim}[fontsize=\relsize{-2}]
cublasSetMatrix(n,n,sizeof(float),hm,n,dm,n);
\end{Verbatim}

is slightly more complicated.  Here we are saying that we are copying
{\bf hm}, an n x n matrix of {\bf float}s on the host, to {\bf dm} on
the host.  The {\bf n} arguments in the last and third-to-last positions
again say that the two matrices each have {\bf n} dimensioned rows.
This seems redundant, but this is needed in cases of matrix tiling,
where the number of rows of a tile would be less than the number of rows
of the matrix as a whole.

The 1s in the call 

\begin{Verbatim}[fontsize=\relsize{-2}]
cublasSetVector(n,sizeof(float),ones,1,drs,1);
\end{Verbatim} 

are needed for similar reasons.  We are saying that in our source vector
{\bf ones}, for example, the elements of interest are spaced 1 elements
apart, i.e. they are contiguous.  But if we wanted our vector to be some
row in a matrix with, say, 500 rows, the elements of any particular row
of interest  would be spaced 500 elements apart, again keeping in mind
that column-major order is assumed.

The actual matrix multiplication is done here:

\begin{Verbatim}[fontsize=\relsize{-2}]
cublasSgemv('n',n,n,1.0,dm,n,drs,1,0.0,drs,1);
\end{Verbatim}

The ``mv'' in ``cublasSgemv'' stands for ``matrix times vector.''  Here
the call says:  no (`n'), we do not want the matrix to be transposed;
the matrix has {\bf n} rows and {\bf n} columns; we wish the matrix to
be multiplied by 1.0 (if 0, the multiplication is not actually
performed, which we could have here); the matrix is at {\bf dm}; the
number of dimensioned rows of the matrix is {\bf n}; the vector is at
{\bf drs}; the elements of the vector are spaced 1 word apart; we wish
the vector to not be multiplied by a scalar (see note above); the
resulting vector will be stored at {\bf drs}, 1 word apart.

Further information is available in the CUBLAS manual. 

\subsection{Thrust}

The Thrust library is usable not only with CUDA but also to general
OpenMP code!  So I've put my coverage of Thrust in a separate chapter,
Chapter \ref{chap:thrust}.

\subsection{CUDPP}

CUDPP is similar to Thrust (though CUDPP was developed earlier) in terms
of operations offered.  It is perhaps less flexible than Thrust, but is
easier to learn and is said to be faster.

(No examples yet, as the author did not have access to a CUDPP system
yet.)

\subsection{CUFFT}

CUFFT does for the Fast Fourier Transform what CUBLAS does for linear
algebra, i.e. it provides CUDA-optimized FFT routines.

\section{Other CUDA Examples in This Book}

There are additional CUDA examples in later sections of this book.
These include:\footnote{If you are reading this presentation on CUDA
separately from the book, the book is at
\url{http://heather.cs.ucdavis.edu/~matloff/158/PLN/ParProcBook.pdf}
}

\begin{itemize}

\item Prof. Richard Edgar's matrix-multiply code, optimized for use of
shared memory, Section \ref{cudamatmult}.

\item Odd/even transposition sort, Section \ref{cudaoddeven}, showing a
typical CUDA pattern for iterative algorithms.

\item Gaussian elimination for linear systems, Section \ref{gauss}.

\end{itemize}

